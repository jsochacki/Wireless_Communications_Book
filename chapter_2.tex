\part{Part 2}
   \chapter{Diversity}
      \section{Deep Fade Events}
         \subsection{High SNR Performance}
            The poor performance of wireless communications systems at high
            SNRs when compared to wired communicaitons systems is due to deep
            fades.
            
            \vspace{10pt}

            \begin{center}
               \begin{tabular}[c]{|w{c}{4cm}|w{c}{4cm}|}
                  \hline
                  \multicolumn{2}{|c|}{Comparison of Wired and wireless
                  communications} \\
                  \hline
                  Wired Channel & Wireless Channel \\
                  \hline
                  $y=x+n$ & $y=hx+n$ \\
                  $BER=Q(\sqrt{SNR})$ &
                  $BER=\frac{1}{2}(1-\sqrt{\frac{SNR}{2+SNR})}$ \\
                 \hline
                 \end{tabular}
            \end{center}

            This can be seen why from the following analysis.

            \subsubsection{Mathematical Simplifications}
               Let us show why by making some mathematical simplifications for
               the BER expression under high SNR.

               \begin{center}
               \begin{align}
                  BER&=\frac{1}{2}\left(1-
                  \sqrt{\frac{SNR}{2+SNR}}\right) \\[10pt]
                     &=\frac{1}{2}\left(1-\frac{1}
                  {\sqrt{1+\frac{2}{SNR}}}\right).
               \end{align}
               \end{center}

               For high SNR we can see that $\frac{2}{SNR}$ is a
               small value and we know that

               \begin{center}
               \begin{align}
                  \frac{1}{\sqrt{1+x_{small}}} = 1-\frac{x_{small}}{2}.
               \end{align}
               \end{center}

               Therefore we can arrive at the approximate equation for BER of a
               wireless channel with a very high SNR.

               \begin{center}
               \begin{align}
                     &\approx \frac{1}{2} \left(1- \left(1-
                     \frac{1}{2}\frac{2}{SNR} \right) \right) \\[10pt]
                     &\approx \frac{1}{2} \frac{1}{SNR}.
               \end{align}
               \end{center}

               \subsubsection{Example 1}
               Compute the bit error rate a wireless communication system at
               $SNR_{dB}=20dB$.

               \begin{center}
               \begin{align}
                  &20dB=10\log_{10}SNR \\[10pt]
                  &\log_{10}SNR=2 \\[10pt]
                  &SNR=10^{2}=10 \\[10pt]
                  &BER=\frac{1}{2SNR} = \frac{1}{2*100} = 50*10^{-4}.
               \end{align}
               \end{center}

               Note that the bit error rate achieved in a wired communication
               system at this same SNR is only $7.8*10^{-4}$.

               \subsubsection{Example 2}
               Compute the SNR in dB of a wireless communication system for a
               $BER=10^{-6}$ 

               \begin{center}
               \begin{align}
                  &10^{-6}=\frac{1}{2SNR} \\[10pt]
                  &SNR=\frac{1}{2*10^{-6}} = \frac{10^{-6}}{2} \\[10pt]
                  &\implies SNR_{dB}
                  =10\log_{10}\left(\frac{10^{-6}}{2}\right)\\[10pt]
                  &=60\text{dB}-3\text{dB}=57\text{dB}.
               \end{align}
               \end{center}

               Note that in a Wired Communication system, the SNR required for
               this same BER is 13.6 dB.  This is because of the desrtuctive
               interference that causes deep fades and is the results of
               multipath propogation.

               \subsubsection{Comparison Of High SNR Approximate Formula}

               \begin{table}[htpb]
               \begin{center}
               \caption{Comparison of high SNR approximate BER formula for
               Wired and Wireless Systems}
               \label{tab:Comparison of high SNR approximate BER formula for
               Wired and Wireless Systems}
               \begin{tabular}[c]{|w{c}{4cm}|w{c}{4cm}|}
                  \hline
                  \multicolumn{2}{|c|}{Comparison of high SNR
                                       approximate BER formula for
                                       Wired and Wireless Systems} \\
                  \hline
                  Wireless System &
                  $BER=\frac{1}{2}\left(1-\sqrt{\frac{SNR}{2+SNR}}\right)$
                  $\approx \frac{1}{2SNR}$ \\
                  \hline
                  Wired System &
                  $BER=Q\left(\sqrt{SNR}\right)$
                  $\approx \exp^{-\frac{SNR}{2}} $\\
                  \hline
               \end{tabular}
               \end{center}
               \end{table}

         \subsection{Probability of a Deep Fade Event}
            Taking a look again at the wireless communications system channel
            model $y=hx+n$.  Let us call $h$ the fading coefficient and $n$ the
            noise.  Remember that we said $\|h\|^{2}P$ is the desired power at
            the receiver and that $\sigma_n^{2}$ is the noise power at the
            receiver.  If we look at the performance of the channel when the
            noise power is greater than the desired power.

            \begin{center}
            \begin{align}
               &\|h\|^{2}P=a^{2}P<\sigma_n^{2} \\[10pt]
               &=a^{2} < \frac{\sigma_n^{2}}{P} \\[10pt]
               \text{Since, } &\frac{P}{\sigma_n^{2}} = SNR \\[10pt]
               &a<\frac{1}{\sqrt{SNR}}
            \end{align}
            \end{center}

            So when $a<\frac{1}{\sqrt{SNR}}$ we say we have a deep fade event
            as we can see that this will create very poor channel
            performance.\\

            \vspace{10pt}
            \textbf{\emph{
            Note, this is a little missleading since we are using the terms SNR
            like it is the receive SNR as one would typically expect when
            rather here it is strictly the transmit SNR so please make sure to
            keep that in mind for the next paragraph and the prior paragraph.
            It is the tx SNR since we are separating the fading coefficient
            from the SNR power meaning that it is not receive SNR.
            This is possibly even incorrect to just forget this section
            "Probability of a deep fade event"\\}}

            \vspace{10pt}

            Since we also know that the probability distribution of a is
            $f_A(a)=2a\exp^{-a^{2}}$, we can just calculate the probability of
            a deep fade event by $P(a<\frac{1}{\sqrt{SNR}})$.  This is just the
            integral of the pdf of a from 0 to a
            $=\int_{0}^{\frac{1}{\sqrt{SNR}}}f_A(a)da$ and if we approximate
            $\exp^{-a^{2}}\approx 1$ then we can say that
            $P(a<\frac{1}{\sqrt{SNR}})=\frac{1}{SNR}$.  Noting also that
            $BER=\frac{1}{2SNR}$ then we can see that $BER\sim \text{The
            Probabililty of a deep fade event}$.  This shows that the poor
            performance of a wireless system at high SNR is due to deep fade
            events that arise from the multipath environment that allows for
            destructive interference.

            \vspace{10pt}

            Adding diversity is one way to reduce the probability of a deep
            fade event.

            \section{Multiple Antenna Systems}

            Below in Figure~\ref{fig:multiple-antenna-system} we can see
            what a multiple antenna system coutld look like.  The system in
            the image below is what is known as a single output multiple
            input (MISO) system as the recevier has multiple inputs and the
            transmitter has a single output.  From a very basic point of
            view, this improves high SNR performance of the wireless system
            as it is much less statistically likely for all channels to
            experience a deep fade event simultaneously. \\

               \begin{figure}[ht]
                  \centering
                  \incfig{chapter_2_multiple-antenna-system}
                  \caption{Multiple Antenna System}
                  \label{fig:multiple-antenna-system}
               \end{figure}

            Using the system in the figure as an example, If we have a
            single transmit antenna and L receive antenna we can say
            that the system has $L^{\text{th}}$ order diversity.
      
            \subsection{MISO System Model}
               Lets now form the system model for the current example
               system.

               \begin{table}[htpb]
               \begin{center}
               \caption{Comparison Of SISO and MISO Systems}
               \label{tab:Comparison Of SISO and MISO Systems}
               \begin{tabular}[c]{|w{c}{4cm}|w{c}{4cm}|}
                  \hline
                  $y=hx+n$ & SISO Wireless System \\
                  \hline
                  $y_1=h_1x+n_1$ & \multirow{4}{*}{MISO Wireless System} \\
                  $y_2=h_2x+n_2$ & \\
                  $\vdots$ & \\
                  $y_\text{L}=h_\text{L}x+n_\text{L}$ & \\
                  \hline
               \end{tabular}
               \end{center}
               \end{table}

               An inspection of the MISO Wireless System makes it
               evident that in fact it is easiest to think of the
               quantities in vector form and represnt the equations as a
               set of vector equations.

               \begin{center}
               \begin{align}
                  \begin{bmatrix} y_1\\ \vdots\\ y_n \end{bmatrix}
                  =\begin{bmatrix} h_1\\ \vdots\\ h_n \end{bmatrix}
                   \begin{bmatrix} x\\ \vdots\\ x \end{bmatrix}
                  +\begin{bmatrix} n_1\\ \vdots\\ n_n \end{bmatrix}
                  \quad = \quad
                  \begin{bmatrix} y_1\\ \vdots\\ y_n \end{bmatrix}
                  =\begin{bmatrix} h_1\\ \vdots\\ h_n \end{bmatrix}
                   x
                  +\begin{bmatrix} n_1\\ \vdots\\ n_n \end{bmatrix}
               \end{align}
               \end{center}

               \vspace{10pt}
               Which is best written in matrix form as
               $\overline{y}=\overline{h}x+\overline{n}$.

               \subsubsection{Analysis of Receive Antenna Diversity System}

                  We will see that the expected value of the noise at each
                  receive antenna is $\sigma_n^{2}$.  This is the power of
                  the noise or in other words it is the noise variance.
                  \begin{center}
                  \begin{align}
                     &\overline{y}=\overline{h}x+\overline{n} \\[10pt]
                     &E \left\{ \left| n_i(k)^{2} \right|  \right\} 
                     =\sigma_n^{2}
                  \end{align}
                  \end{center}

                    \paragraph{signal Detection}
                    $y_1, y_2, \cdots , y_L $ are the singals received at
                    the L receive antennas.  We can combine these receive
                    signals in the following manner $w_1^{*}y_1+w_2^{*}y_2
                    +\cdots+w_\text{L}^{*}y_\text{L}$
                    then we are combining them in a weighted manner which
                    we will call beamforming.  We will refer to
                    $\overline{W}$ as the beamformer or beamforming matrix
                    (when in matrix form).

                    \begin{center}
                    \begin{align}
                       \overline{W}=&\begin{bmatrix} W_1\\
                       \vdots\\ W_n \end{bmatrix} \\[10pt]
                       \begin{bmatrix} W_1^{*}& \cdots&
                       W_n^{*} \end{bmatrix}
                       &\begin{bmatrix} y_1 \\ \vdots \\ y_n \end{bmatrix}
                       =\overline{W}^{H}\overline{y}
                    \end{align}
                    \end{center}

                    Since
                    $\overline{W}^{H}\overline{y}$ and
                    $\overline{y}=\overline{h}x+\overline{n}$
                    , our beamformer output is
                    $\overline{W}^{H}\left(\overline{h}x
                    +\overline{n}\right)$.  Expanding that we can see that
                    we have $\overline{W}^{H}\overline{h}x
                    +\overline{W}^{H}\overline{n}$.  We will call
                    $\overline{W}^{H}\overline{h}x$ the signal component
                    and $\overline{W}^{H}\overline{n}$ the noise component.
                    Therefore we have

                    \begin{center}
                    \begin{align}
                    SNR=\frac{\text{Signal Power}}{\text{Noise Power}}
                    =\frac{\left|\overline{W}^{H}h\right|^{2}P}
                    {E \left\{\left|\overline{W}^{H}\overline{n}\right|^{2}
                    \right\}}.
                    \end{align}
                    \end{center}

